\documentclass{article}

\usepackage[francais]{babel}
\usepackage[T1]{fontenc}
\usepackage[latin1]{inputenc}
\usepackage{amsmath,amssymb,mathrsfs}

\newcommand{\fonca}{x \ln x}
\newcommand{\foncb}{\ln(1+2x)}
\newcommand{\foncc}{x \ln x}
\newcommand{\foncd}{\sqrt{x^2+3x}\ln(x^2)}
\newcommand{\fonce}{\frac{1}{x+1}}
\newcommand{\foncf}{\ln(1+\frac{1}{x})}
\newcommand{\foncg}{}
\newcommand{\fonch}{}

\begin{document}

%SWAMINATHAN Gurukandhamoorthi
%Etude locale des fonctions

\'Etude locale des fonctions\newline

a) $\fonca$ et $\foncb$ au voisinage de $0$\newline

\begin{eqnarray*}
\lim_{x \rightarrow 0}\frac{\fonca}{\foncb}&=&-\infty\\
\ln(1+u) &\underset{0}{\sim}& u \\
 %comportement de ln
 \mbox{Donc, } \foncb &\underset{0}{\sim}& 2x\\
%$$ \mbox{D'o�: } \frac{\fonca}{\foncb} \sim \frac{\ln x}{2} \mathop{\mathop{\rightarrow}_{x\rightarrow 0}}_{x>0}-\infty$$
  \mbox{D'o�: } \frac{\fonca}{\foncb} &\sim& \frac{\ln x}{2} \underset{x>0}{\underset{x\rightarrow 0}{\rightarrow}}-\infty\\
  \mbox{Donc, } \frac{\foncb}{\fonca}&\rightarrow& 0\\
   \mbox{Donc, } \foncb&=&o(\fonca) \mbox{ au } v(0)\\
\end{eqnarray*}
 b) comparer $\foncc$ et $\foncd$
$$\lim_{x \rightarrow +\infty}\frac{\foncd}{\foncc}$$
\begin{eqnarray*}
                              \foncd&=& 2\sqrt{x^2+3x}\ln(x)\\
                                    &=& 2x\sqrt{1+\frac{3}{x}}\ln(x)\\
  \mbox{Donc, }\frac{\foncd}{\foncc}&=& 2\sqrt{1+\frac{3}{x}}\\
\mbox{Donc le quotient est born� au }v(+\infty)\\
  \mbox{Donc, } \foncd&=&o(\foncc) \mbox{ au } v(+\infty)\\
\frac{\foncc}{\foncd} &\underset{x\rightarrow +\infty}{\rightarrow}& \frac{1}{2}\\
 \mbox{Donc, } \foncc&=&o(\foncd) \mbox{ au } v(+\infty)\\
\end{eqnarray*}

c) $\fonce$ et $\foncf$ au voisinage de $-1$\newline
$$\lim_{x \rightarrow +\infty}\frac{\foncf}{\fonce}$$
On pose $u=1+x$\\
$\fonce=\frac{1}{u}$ et $\foncf=\ln(1+\frac{1}{u-1})=\ln(\frac{u}{u-1})$ $u\rightarrow0^-$
$$\lim_{u \rightarrow 0}\frac{\ln(\frac{u}{u-1})}{\frac{1}{u}}=0$$
$$u\ln(\frac{u}{u-1})=u(\ln(-u)-\ln(1-u))\underset{u\rightarrow 0^-}{\rightarrow}0$$
\begin{eqnarray*}
\mbox{Donc, }\ln(\frac{u}{u-1})&=&o(\frac{1}{u}) \mbox{ au } v(0)\\
\mbox{Donc, } \foncf&=&o(\fonce) \mbox{ au } v(-1)\\
\end{eqnarray*}
\end{document}


