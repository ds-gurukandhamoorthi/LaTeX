\documentclass{article}
\usepackage[francais]{babel}
\usepackage[T1]{fontenc}
\usepackage[latin1]{inputenc}
\usepackage{amsmath,amssymb,mathrsfs}
\begin{document}
\textbf{Solution de la devinette}\newline

D'une part, cette surface est un carr�.\newline
Le c�t� du carr� mesure $1+2+3+...+n=\frac{n(n+1)}{2}$.\newline
L'aire du carr� vaut donc $(\frac{n(n+1)}{2})^2$.\newline
Passage de n � n+1:
\begin{eqnarray*}
\frac{n(n+1)}{2}+(n+1)&=&\frac{n(n+1)+2(n+1)}{2}\\
                      &=&\frac{(n+1)(n+2)}{2}\\
\end{eqnarray*}
(On fait la r�currence sur le c�t� du carr�, puis il suffit de �lever cette quantit� au carr�.)\newline

D'autre part, on sait que $n^3=n(n^2)$.\newline
Remarque: Un cube peut �tre d�coup� en n surfaces. Visualiser les cubes qu'utlisent les chimistes pour r�parition des atomes dans l'espace peut �tre d'une grande aide. Ces cubes contiennent $n^3$ boules (atomes) li�es par des tiges(liens atomiques).\newline
\newline
Passage de n � n+1:
\begin{eqnarray*}
2\frac{n(n+1)}{2}(n+1)+(n+1)^2&=&(n+1)^2(n+1)\\
                              &=&(n+1)^3\\ 
\end{eqnarray*}
En effet, le nombre de jetons qu'on ajoute peut �tre vu comme deux rectangles et un carr�($(n+1)^2$). Ces deux rectangles ont m�me surface : l'un est plac� verticalement, l'autre horizontalement. Leurs c�t�s valent $(n+1)$ et $\frac{n(n+1)}{2}$ (surface du carr� d'ordre n).

Initialisations:\newline
$$1=(\frac{1(1+1)}{2})^2$$
$$1=1^3$$
\newline
$$\mbox{D'o�: }\sum_{k=1}^n{k^3}=(\frac{n(n+1)}{2})^2$$
 \newline
 \newline
  \newline
 \newline
 \texttt{Guru}
\end{document}